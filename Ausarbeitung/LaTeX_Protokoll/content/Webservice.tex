\subsection{Was ist ein Webservice?}
Ein Webservice erm\"{o}glicht eine Maschine-zu-Maschine Kommunikation mittels
HTTP oder HTTPS \"{u}ber das Internet. Auf einem entfernten Computer werden Funktionen
ausgef\"{u}hrt. Der Webservice wird mittels URI (Uniform Resource Identifier)
eindeutig identifiziert. Die Kommunikation kann HTTP, XML oder JSON basiert sein.

\subsection{Welchen Webservice nutzt Amazons Alexa?}
Alexa arbeitet mit dem Amazon Web Service (AWS). AWS ist ein Tochterunternehmen von Amazon.
AWS ist ein Cloud Computing Anbieter. Ein Nutzer von AWS bekommt einen virtuellen Server, welcher
auf einem echten Server innerhalb einer Serverfarm l\"{a}uft.
F\"{u}r diesen virutellen Server bietet AWS diverse Dienste an.
Die f\"{u}r den Voice Assistant ben\"{o}tigten sind dabei folgende:
\subsubsection{Lambda}
Der AWS Dienst Lambda f\"{u}hrt beliebigen Code auf dem virtuellen Server aus.
Meistens wird Lambda dazu verwendet eine Anfrage zu verarbeiten.
In dieser Anfrage wird eine Lambda Funktion aufgerufen, ein von einem Programmierer
entwickeltes Programm. Die Lambda Funktion generiert eine Antwort, meist im JSON-Format.
Zurzeit unterst\"{u}tzte Programmiersprachen mit den entsprechenden
Bibliotheken sind: Java, Node.js, Python und C\#.
In zusammenarbeit mit einem Voice Assistenten wird dem AWS Dienst Lambda eine Anfrage im
JSON-Format \"{u}bermittelt, in dieser steht die Lambda Funktion welche ausgef\"{u}hrt werden soll
mit eventuellen Parametern. Der hinterlegte Programmcode (Logic Code) wird ausgef\"{u}hrt.
Die Antwort im JSON-Format sendet AWS zur\"{u}ck an den Voice Assistant.
Dieser wandelt die Antwort mittels TTS in Sprache um und Antwortet dem Nutzer auf seine Anfrage.
\subsubsection{Alexa Voice Service (AVS)}
AVS versteht nat\"{u}rliche Sprache und kann diese in Text umwandeln.
Umgekehrt kann AVS Text wieder in Sprache umwandeln.
Die aktuellste Version des AVS wird von Alexa auf das Voice Assistant Ger\"{a}t heruntergeladen.
Der Voice Assistant wird dann lokal die Sprache in Text umwandeln und an den AWS senden.
Da AVS ein Teil der Amazon Web Services ist wird die Spracherkennung mittels Machine Learning
stettig verbessert, d.h. viele Ger\"{a}te senden erfolgreiche oder fehlgeschlagene Anfragen an AWS,
um die Spracherkennung zu verbessern. Durch das Alexa Skill Kit kann ein Entwickler dem AVS
zeigen welche Reaktion der Entwickler bei einem bestimmten Stichwort oder Satz erwartet.
Dabei muss nahezu jede m\"{o}gliche Variation des Satzes bedacht sein, damit entsprechend
darauf reagiert werden kann.

\subsection{Welchen Webservice nutzt Google Assistant?}
Die Google Home Ger\"{a}te integrieren den Google Assistant. Dieser kann mittels Dialogflow
(fr\"{u}her APi.ai) erweitert werden, wie vorher bereits mit mobilen Ger\"{a}ten oder dem Google Chrome
Browser m\"{o}glich. Dialogflow wird durch eine Node.js Node auf einem eigenen Server oder z.B.
einem Server des Google Cloud Projects ausgef\"{u}hrt. Der Entwickler hat einen Agent (Anwednung)
entwickelt, welcher auf die Anfrage des Nutzers reagiert.
